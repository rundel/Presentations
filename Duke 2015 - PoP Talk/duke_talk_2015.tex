\documentclass[t]{beamer}\usepackage[]{graphicx}\usepackage[]{color}
%% maxwidth is the original width if it is less than linewidth
%% otherwise use linewidth (to make sure the graphics do not exceed the margin)
\makeatletter
\def\maxwidth{ %
  \ifdim\Gin@nat@width>\linewidth
    \linewidth
  \else
    \Gin@nat@width
  \fi
}
\makeatother

\definecolor{fgcolor}{rgb}{0.345, 0.345, 0.345}
\newcommand{\hlnum}[1]{\textcolor[rgb]{0.686,0.059,0.569}{#1}}%
\newcommand{\hlstr}[1]{\textcolor[rgb]{0.192,0.494,0.8}{#1}}%
\newcommand{\hlcom}[1]{\textcolor[rgb]{0.678,0.584,0.686}{\textit{#1}}}%
\newcommand{\hlopt}[1]{\textcolor[rgb]{0,0,0}{#1}}%
\newcommand{\hlstd}[1]{\textcolor[rgb]{0.345,0.345,0.345}{#1}}%
\newcommand{\hlkwa}[1]{\textcolor[rgb]{0.161,0.373,0.58}{\textbf{#1}}}%
\newcommand{\hlkwb}[1]{\textcolor[rgb]{0.69,0.353,0.396}{#1}}%
\newcommand{\hlkwc}[1]{\textcolor[rgb]{0.333,0.667,0.333}{#1}}%
\newcommand{\hlkwd}[1]{\textcolor[rgb]{0.737,0.353,0.396}{\textbf{#1}}}%

\usepackage{framed}
\makeatletter
\newenvironment{kframe}{%
 \def\at@end@of@kframe{}%
 \ifinner\ifhmode%
  \def\at@end@of@kframe{\end{minipage}}%
  \begin{minipage}{\columnwidth}%
 \fi\fi%
 \def\FrameCommand##1{\hskip\@totalleftmargin \hskip-\fboxsep
 \colorbox{shadecolor}{##1}\hskip-\fboxsep
     % There is no \\@totalrightmargin, so:
     \hskip-\linewidth \hskip-\@totalleftmargin \hskip\columnwidth}%
 \MakeFramed {\advance\hsize-\width
   \@totalleftmargin\z@ \linewidth\hsize
   \@setminipage}}%
 {\par\unskip\endMakeFramed%
 \at@end@of@kframe}
\makeatother

\definecolor{shadecolor}{rgb}{.97, .97, .97}
\definecolor{messagecolor}{rgb}{0, 0, 0}
\definecolor{warningcolor}{rgb}{1, 0, 1}
\definecolor{errorcolor}{rgb}{1, 0, 0}
\newenvironment{knitrout}{}{} % an empty environment to be redefined in TeX

\usepackage{alltt}

\usetheme{Madrid}
\usecolortheme{seahorse}

\usepackage{geometry}
\usepackage{graphicx}
\usepackage{amssymb}
\usepackage{epstopdf}
\usepackage{amsmath}  	% this permits text in eqnarray among other benefits
\usepackage{color}          	% gives color options
\usepackage{url}		% produces hyperlinks
\usepackage[english]{babel}
\usepackage[latin1]{inputenc}
\usepackage{colortbl}	% allows for color usage in tables
\usepackage{multirow}	% allows for rows that span multiple rows in tables
\usepackage{xcolor}		% this package has a variety of color options
\usepackage{calc}
\usepackage{multicol}

\setbeamertemplate{navigation symbols}{}

%User defined colors: See colors section
\xdefinecolor{oiBlue}{rgb}{0.15, 0.35, 0.55}
\xdefinecolor{gray}{rgb}{0.5, 0.5, 0.5}
\xdefinecolor{darkGray}{rgb}{0.3, 0.3, 0.3}
\xdefinecolor{darkerGray}{rgb}{0.2, 0.2, 0.2}
\xdefinecolor{rubineRed}{rgb}{0.89,0,0.30}
\xdefinecolor{linkCol}{rgb}{0.11,0.49,0.95}	
\xdefinecolor{irishGreen}{rgb}{0,0.60,0}	
\xdefinecolor{darkturquoise}{rgb}{0.44, 0.58, 0.86}
\definecolor{lightGreen}{rgb}{0.533,0.765,0.42}

\setbeamercolor*{palette primary}{fg=white,bg= oiBlue!70}
\setbeamercolor*{palette secondary}{fg=black,bg= oiBlue!20!white}
\setbeamercolor*{palette tertiary}{fg=white,bg= oiBlue!80!black!90}
\setbeamercolor*{palette quaternary}{fg=white,bg= oiBlue}

\setbeamercolor{structure}{fg= oiBlue}
\setbeamercolor{frametitle}{bg= oiBlue!70}

\setbeamercolor{disc body}{bg=oiBlue!20!white!80,fg=oiBlue!80!black!90}
\setbeamercolor{disc title}{bg=oiBlue!40!white!60,fg=oiBlue!70!black!100}


\setbeamertemplate{blocks}[shadow=false]


\newcommand{\removepagenumbers}{% 
  \setbeamertemplate{footline}{
    %
    \begin{beamercolorbox}[colsep=1.5pt]{upper separation line foot}
    \end{beamercolorbox}
    \begin{beamercolorbox}[ht=2.5ex,dp=1.125ex,%
      leftskip=.3cm,rightskip=.3cm plus1fil]{author in head/foot}%
      \leavevmode{\usebeamerfont{author in head/foot}\insertshortauthor}%
%      \hfill%
%      {\usebeamerfont{author in head/foot}\usebeamercolor[fg]{institute in head/foot}\insertshortinstitute}%
    \end{beamercolorbox}%
    \begin{beamercolorbox}[ht=2.5ex,dp=1.125ex,%
      leftskip=.3cm,rightskip=.3cm plus1fil]{title in head/foot}%
      {\usebeamerfont{title in head/foot}\insertshorttitle}%
      \hfill%
      {\usebeamerfont{author in head/foot}\usebeamercolor[fg]{institute in head/foot}\insertshortinstitute}%
    \end{beamercolorbox}%
    \begin{beamercolorbox}[colsep=1.5pt]{lower separation line foot}
    \end{beamercolorbox}
    }
} 


\newcommand{\disc}[2]{
\begin{beamerboxesrounded}[shadow = true, lower = disc body, upper = disc title]{#1}
#2
\end{beamerboxesrounded}
}


\AtBeginSection[] 
{ 
  \addtocounter{framenumber}{-1} 
  % 
  {\removepagenumbers 
    \begin{frame}<beamer> 
    \tableofcontents[currentsection] 
  \end{frame} 
  } 
} 

\usepackage{bm}
\usepackage{isotope}
\usepackage{appendixnumberbeamer}
\usepackage{dsfont}

\newcommand{\PM}{$\text{PM}_{2.5}$ }


\definecolor{redhl}{rgb}{0.98,0.29,0.28}
\definecolor{yellowhl}{rgb}{0.98,0.87,0.28}
\newcommand{\hlr}[1]{\fcolorbox{redhl}{white}{$\displaystyle #1$}}
\newcommand{\hly}[1]{\fcolorbox{yellowhl}{white}{$\displaystyle #1$}}

\newcommand{\vvfill}{\vskip0pt plus 1filll}

\title[GPU GPs]{GPUs and the computational efficiency\\ of Gaussian process based models}
\author{Colin Rundel}
\date{April 15, 2015}
\institute[Duke]{Duke University}
\IfFileExists{upquote.sty}{\usepackage{upquote}}{}
\begin{document}

\begin{frame}[plain]
\titlepage
\end{frame}



%==================================================================================================

\section{Background}
\addtocounter{framenumber}{-1} 

%==================================================================================================

\begin{frame}
\frametitle{The problem with GPs ...}

Unless you are lucky (or clever), Gaussian process models are difficult to scale to large problems. For a Gaussian process $\mathcal{N}(\bm{\mu},\bm{\Sigma})$:

\pause \vspace{3mm}

\only<2->{
  Want a sample?
  %
  \twocol{0.8}{0.2}{
    \only<3>{
      \[ \bm{\mu} + \text{Chol}(\bm{\Sigma}) \times \bm{Z} \text{ with } Z_i \sim \mathcal{N}(0,1)\]
    }
    \only<4->{
      \[ \bm{\mu} + \hlr{\text{Chol}(\bm{\Sigma})} \times \bm{Z} \text{ with } Z_i \sim \mathcal{N}(0,1) \]
    }
  }{
    \vspace{-4mm}\only<4->{\[\color{redhl}{\mathcal{O}\left(n^3\right)}\]}
  }
}

\only<5->{
Evaluate the (log) likelihood? \pause
%
\twocol{0.8}{0.2}{
\only<6>{
\[ -\frac{1}{2} \log |\Sigma| - \frac{1}{2} (\bm{x}-\bm{\mu})' \bm{\Sigma}^{-1} (\bm{x}-\bm{\mu}) - \frac{n}{2}\log 2\pi \]
}
\only<7->{
\[ -\frac{1}{2} \log \hlr{|\Sigma|} - \frac{1}{2} (\bm{x}-\bm{\mu})' \hlr{\bm{\Sigma}^{-1}} (\bm{x}-\bm{\mu}) - \frac{n}{2}\log 2\pi\]
}
}{\vspace{-4mm}\only<7->{\[\color{redhl}{\mathcal{O}\left(n^3\right)}\]}}
}

\only<8->{
Update covariance parameter?
%
\twocol{0.8}{0.2}{
\only<9>{
\[ \{\Sigma\}_{ij} = \sigma^2 \exp(-\{d\}_{ij}\phi) \]
}
\only<10->{
\[ \hly{\{\Sigma\}_{ij} = \sigma^2 \exp(-\{d\}_{ij}\phi)}\]
}
}{\vspace{-4mm}\only<10->{\[\color{yellowhl}{\mathcal{O}\left(n^2\right)}\]}}
}

\end{frame}


%==================================================================================================

\begin{frame}[c]
\frametitle{A simple guide to computational complexity}

{\Large \begin{center}
\vfill
Linear complexity? \pause- Go for it \pause

\vspace{15mm}

Quadratic complexity? \pause- Pray \pause

\vspace{15mm}

Cubic complexity? \pause- Give up

\vfill
\end{center} }
\end{frame}

%==================================================================================================

\begin{frame}
\frametitle{Improving Cholesky}
    
\vspace{-8mm}

\begin{center}
\begin{knitrout}\footnotesize
\definecolor{shadecolor}{rgb}{0.969, 0.969, 0.969}\color{fgcolor}

{\centering \includegraphics[width=\textwidth]{plots/unnamed-chunk-1-1} 

}



\end{knitrout}
\end{center}


\end{frame}

%==================================================================================================

\begin{frame}
\frametitle{Tools and Optimization}

\vfill

\begin{center}
\includegraphics[width=0.9\textwidth]{figs/diagram.pdf}
\end{center}

\vfill

Regardless of tools or workflow, measuring / profiling performance is critical.

\vfill

\end{frame}

%==================================================================================================

\section{Migratory Bird Spatial Assignment Model}

%==================================================================================================

\begin{frame}
\frametitle{Background}

Using intrinsic markers (genetic and isotopic signals) for the purpose of inferring migratory connectivity.

\vspace{2mm}

\begin{itemize}
\item Existing methods are too coarse for most applications
\vspace{2mm}
\item Large amounts of data are available ( \textgreater{}150,000 feather samples from \textgreater{}500 species)
\vspace{2mm}
\item Genetic assignment methods are based on Wasser, et al. (2004)
\vspace{2mm}
\item Isotopic assignment methods are based on Wunder, et al. (2005)
\end{itemize}

\end{frame}

%==================================================================================================

\begin{frame}[t]
\frametitle{Data - DNA microsatellites and $\delta \isotope[2]{H}$}

\begin{columns}[t]
\column{0.5\textwidth}
Hermit Thrush (\textit{Catharus guttatus}) \\
\vspace{2mm}
\begin{itemize}
\item 138 individuals
\item 14 locations
\item 6 loci
\item 9-27 alleles / locus
\end{itemize}
\column{0.5\textwidth}
Wilson's Warbler (\textit{Wilsonia pusilla}) \\
\vspace{2mm}
\begin{itemize}
\item 163 individuals
\item 8 locations
\item 9 loci
\item 15-31 alleles / locus
\end{itemize}

\end{columns}

~\\
~\\

\begin{columns}[t]
\column{0.5\textwidth}
\begin{center}
\includegraphics[width=0.65\textwidth]{figs/hermit_thrush.jpeg}
\end{center}
\column{0.5\textwidth}
\begin{center}
\includegraphics[width=0.65\textwidth]{figs/wilsons_warbler.jpeg}
\end{center}
\end{columns}


\end{frame}

%==================================================================================================

\begin{frame}
\frametitle{Allele Frequency Model}

For the allele $i$, from locus $l$, at location $k$

\begin{align*}
\bm{y}_{\cdot l k}|\bm{\Theta} &\sim \text{Mult}\left(\textstyle\sum_i y_{ilk},\: \bm{f}_{\cdot l k}\right) \\
\\
f_{ilk} &= \frac{\exp(\Theta_{ilk})}{\sum_i \exp(\Theta_{ilk})} \\
\\
\bm{\Theta}_{il}|\bm{\alpha},\bm{\mu} &\sim \mathcal{N}( \bm{\mu}_{il},\, \bm{\Sigma_{}}) \\
\end{align*}

\[ \left\{\Sigma\right\}_{ij} = \alpha_0 \, \exp \Big(-(\{d\}_{ij}/\alpha_1)^{\alpha_2} \Big) + \alpha_3 \, \mathds{1}_{i=j} \]

\end{frame}


%==================================================================================================


\begin{frame}
\frametitle{Genetic Assignment  Model}

Assignment model using Hardy-Weinberg equilibrium allowing for genotyping ($\delta$) and single amplification ($\gamma$) errors.

\begin{align*}
P(S_G|\bm{f},k) &= \prod_l P(i_l, j_l | \bm{f},k) \\
\\
P(i_l, j_l | \bm{f},k) &= 
\begin{cases}
\gamma P(i_l|\bm{f},k) + (1-\gamma)P(i_l|\bm{\tilde f},k)^2 & \text{if $i=j$} \vspace{2mm} \\
(1-\gamma) P(i_l|\bm{f},k) P(j_l|\bm{f},k)      & \text{if $i \ne j$}
\end{cases} \\
\\
P(i_l|\bm{f},k) &= (1-\delta) f_{lik} + \delta / m_l
\end{align*}

\end{frame}

%==================================================================================================

% \begin{frame}
% \frametitle{Isotope Model}
% 
% \[ S_I | k,\bm{\tilde p}, \omega, \rho, \tau^2 \sim \text{N}(\omega+\rho \, \tilde{p}_k, \tau^2) \]
%  
% \vfill
% 
% \begin{center}
% \includegraphics[width=0.5\textwidth]{figs/isoscape.pdf}
% \end{center}
% 
% \vfill
% 
% \end{frame}

%==================================================================================================

\begin{frame}
\frametitle{Combined Model}

\vfill

\begin{center}

Genetic \qquad\qquad\qquad\quad
Isotopic \qquad\qquad\qquad\quad
Combined

\end{center}

\begin{center}
\includegraphics[width=\textwidth]{figs/hermit_maps.pdf}
\end{center}

\vfill

\end{frame}

%==================================================================================================

%\begin{frame}
%\frametitle{Model Assessment}
%
%\vspace{-3mm}
%
%\begin{center}
%\includegraphics[width=\textwidth]{figs/ROCs.pdf}
%\end{center}
%
%\end{frame}

%==================================================================================================

\begin{frame}
\frametitle{Migratory Connectivity}

\vspace{-3mm}

\begin{center}
\includegraphics[width=0.9\textwidth]{figs/wintering.png}
\end{center}

\end{frame}

%==================================================================================================

\begin{frame}
\frametitle{Implementation}

Model fitting is done via MCMC (MH within Gibbs) \\
\begin{itemize} \addtolength{\itemsep}{3mm}
\item Original implementation in pure C++ with minimal dependencies (Wasser, et al. (2004))
\item Rewritten using R / C++ via Rcpp(Armadillo) 
\begin{itemize}
\item Code closer to matrix notation (and R)
\item Transparent use of high performance LAPACK implementations
\item R Package - isoscatR - \url{https://github.com/rundel/isoscatR}
\end{itemize}
\item Model fitting performance is quite good
\begin{itemize}
  \item 300,000 iterations in $\sim 5.5$ minutes
\end{itemize}
\item Bottleneck in drawing posterior predictive samples
\begin{itemize}
  \item1,000 iterations in $\sim 30$ minutes
\end{itemize}
\end{itemize}

\end{frame}

%==================================================================================================

\begin{frame}
\frametitle{Prediction details}

Why is the prediction slow? \pause 

~\\

Predicting allele frequencies for Hermit thrush at 3318 novel locations.\\ \pause

~\\

To do so we sample from:
\[ \bm{\Theta}_p | \bm{\Theta}_m \sim \mathcal{N}(\bm{\mu}_p+\bm\Sigma_{pm}\bm\Sigma_{m}^{-1}(\bm{\Theta}_m-\bm\mu_m),\: \bm\Sigma_{p}-\bm\Sigma_{pm}\bm\Sigma_{m}^{-1}\bm\Sigma_{mp}) \]

\pause

\vspace{-3mm}

\begin{columns}
\column{0.15\textwidth}
\column{0.70\textwidth}
\begin{block}{Algorithm steps}
\begin{enumerate}
\item Calculate $\bm\Sigma_{pm}$, $\bm\Sigma_{p}$, and $\bm\Sigma_{p}-\bm\Sigma_{pm}\bm\Sigma_{m}^{-1}\bm\Sigma_{mp}$
\item Calculate $\text{Chol}(\bm\Sigma_{p}-\bm\Sigma_{pm}\bm\Sigma_{m}^{-1}\bm\Sigma_{mp})$
\item Sample from MVN
\item Calculate allele frequencies
\end{enumerate}
\end{block}
\column{0.2\textwidth}
\end{columns}
\end{frame}

%==================================================================================================

\begin{frame}
\frametitle{Posterior predictive sampling timings}

% Performance (CPU) :
% =============================================
% Step 1: 1.08 (0.0106)
% Step 2: 0.000174 (5.15e-06)
% Step 3: 0.467 (0.00171)
% Step 4: 0.0491 (0.000725)
% Step 5: 0.129 (0.000273)
% Step 6: 0.00654 (0.0338)


% Performance (CPU+GPU) :
% =============================================
% Step 1: 0.0462 (0.000196)
% Step 2: 3.43e-05 (3.02e-06)
% Step 3: 0.208 (0.000229)
% Step 4: 0.0525 (0.000227)
% Step 5: 0.127 (0.00188)
% Step 6: 0.032 (0.231)

\vspace{-7mm}

\begin{center}
\renewcommand*{\arraystretch}{1.5}
\begin{tabular}{rl|c|c|c}
& Step                                    & CPU (secs)  & CPU+GPU (secs)  & Rel. Performance \\
\hline
1. & Covariances                          & 1.080       & 0.046           & 23.0 \\
2. & Cholesky                             & 0.467       & 0.208           & 2.3 \\
3. & Sample                               & 0.049       & 0.052           & 0.9 \\
4. & Allele Freq                          & 0.129       & 0.127           & 1.0 \\
\hline 
   & Total                                & 1.732       & 0.465           & 3.7 \\
\end{tabular}

\end{center}

\vspace{3mm}
\only<2->{
\begin{columns}[t]

\column{0.42\textwidth}
Total run time: \\ \vspace{2mm}
\begin{itemize}
\item CPU - 28.9 minutes
\item CPU+GPU - 7.8 minutes
\end{itemize}

\column{0.58\textwidth}
\only<3>{
~\\ \vspace{5mm}
$\times \text{ CV runs} \; \left[\begin{array}{l}
166 \text{ for Hermit Thrush} \\
179 \text{ for Wilson's Warbler}
\end{array}\right]$
}
\end{columns}
}
\end{frame}

%==================================================================================================

\begin{frame}
\frametitle{Lessons}

\begin{itemize}    

\item Relatively small changes in one function resulted in 3 - 4x improvement

\begin{itemize}
\vspace{2mm} \item Cross validation results in two days instead of a week

\vspace{2mm} \item 1-2 weeks of implementation, 1 week of tweaking / testing

\vspace{2mm} \item Started with Cholesky, other optimizations followed
\end{itemize}

\vspace{7mm} \pause

\item Issues

\begin{itemize}
\vspace{2mm} \item External library dependency makes package development \\ (much) more complicated

\vspace{2mm} \item Additional code verbosity and complexity

\end{itemize}
\end{itemize}

\end{frame}

%==================================================================================================

\begin{frame}
\frametitle{Improving Covariance Calculations}
    
Covariance is assumed to be stationary and isotropic
\begin{itemize}
\item Elements of the covariance matrix can be calculated independently
\item Small scale ``embarrassingly parallel''
\item Implementation is straight forward \\ (if we don't worry about things like symmetry)
\end{itemize}

\vfill
\begin{center}
\fbox{
\includegraphics[width=0.95\textwidth]{figs/pow_exp_kernel.png}
}
\end{center}
\vfill

\end{frame}




%==================================================================================================

\begin{frame}
\frametitle{Building core tools}

Common set of (expensive) tasks for GP models

\vspace{1mm}

\begin{itemize}
\item Covariance calculation
\item Cholesky of Cov. 
\item Inverse of Cov. 
\end{itemize}

\vspace{2mm}

Goal is to make performing these tasks on a GPU as painless as possible and allow interoperability with GPU (magma, CUBLAS) and CPU (Armadillo) libraries.

\begin{itemize}
\item GPU matrix class
\item Modern resource management (RAII, move semantics)
\item Simple translation between GPU and CPU memory
\end{itemize}

\vfill

R Package - RcppGP - \url{https://github.com/rundel/RcppGP}

\end{frame}

%==================================================================================================

\begin{frame}[label=code]
\frametitle{CPU vs GPU code}

\vfill

\begin{center}
\fbox{\includegraphics[width=0.9\textwidth]{figs/CPU.png}}

\vspace{10mm}

\fbox{\includegraphics[width=0.9\textwidth]{figs/GPU.png}}
\end{center}

\vfill


\hyperlink{pm_lessons}{\beamerbutton{Back}}
\end{frame}


%==================================================================================================

\section{Speciated PM$_{2.5}$ Modeling}

%==================================================================================================

\begin{frame}
\frametitle{Background}
    
Fine particulate matter (\PM{}) is an EPA regulated air pollutant linked to a variety of adverse health effects

\begin{itemize}
  \vspace{2mm} \item Classified based on particle size ($<2.5$ $\mu$m diameter)
  \vspace{2mm} \item Major species: Sulfate, Nitrate, Ammonium, Soil, Carbon.
  \vspace{2mm} \item Minor species: trace elements (K, Mg, Ca), heavy metals (Cu, Fe), etc.
  \vspace{2mm} \item Complex spatio-temporal dependence between species
\end{itemize}


\end{frame}

%==================================================================================================

\begin{frame}
\frametitle{Data}

Speciated \PM Sources
\begin{itemize}
  \item Chemical Speciation Network (CSN) - 221 stations
  \item Interagency Monitoring of Protected Visual Environments (IMPROVE) - 172 stations
\end{itemize}

\vspace{2mm}

Total \PM Sources
\begin{itemize}
  \item Federal Reference Method (FRM) - 949 stations
\end{itemize}

\vspace{2mm}

Model Output
\begin{itemize}
  \item Community Multi-scale Air Quality (CMAQ) - 12 km grid
\end{itemize}

\vspace{5mm}

Data Issues
\begin{itemize}
  \item Monitoring frequency
  \item Total vs Sum of Species
\end{itemize}

\end{frame}


%==================================================================================================

\begin{frame}
\frametitle{Species Model Details}

For the 5 major species (Sulfate, Nitrate, Ammonium, Soil, Carbon) and the two networks (CSN, IMPROVE):
%
\begin{align*}
C_t^i(\bm{s}) &= Z_t^i(\bm{s}) + \epsilon_{C,t}^i(\bm{s}) \\
I_t^i(\bm{s}) &= Z_t^i(\bm{s}) + \epsilon_{I,t}^i(\bm{s})
\end{align*}

where $Z_t^i(\bm{s})$ are the latent ``true'' concentrations of species $i$ at time $t$ and locations $\bm{s}$, and is given by
%
\begin{align*}
{Z}_t^i(\bm{s}) &= \max{}\left(0,~\widetilde{Z}_t^i(\bm{s})\right) \\
\widetilde{Z}_t^i(\bm{s}) &= \beta_{0,t}^i +\beta_{0,t}^i(\bm{s}) + \beta_{1,t}^i \: Q_t^i(B_{\bm{s}})  
\end{align*}


\end{frame}

%==================================================================================================

\begin{frame}
\frametitle{Total \PM Model Details}

For total \PM from the three networks (CSN, IMPROVE, FRM):
%
\begin{align*}
C_t^{tot}(\bm{s}) &= Z_t^{tot}(\bm{s}) + \epsilon_{C,t}^{tot}(\bm{s}) \\
I_t^{tot}(\bm{s}) &= Z_t^{tot}(\bm{s}) + \epsilon_{I,t}^{tot}(\bm{s}) \\
F_t^{tot}(\bm{s}) &= Z_t^{tot}(\bm{s}) + \epsilon_{F,t}^{tot}(\bm{s})
\end{align*}

where $Z_t^{tot}(\bm{s})$ are the latent ``true'' concentration of total \PM at time $t$ and locations $\bm{s}$, which is given by the sum of the major species and the ``other'' species concentrations.
%
\[ Z^{tot}_{t}(\bm{s}) = \sum_{i=1}^{5} Z^i_t(\bm{s}) + Z^{o}_{t}(\bm{s}) \]

{\footnotesize
\[
{Z}_t^o(s) = \max{}\left(0,~\widetilde{Z}_t^o(s)\right) \qquad
\widetilde{Z}_t^{o}(\bm{s}) = \beta_{0,t}^o +\beta_{0,t}^o(\bm{s}) + \beta_{1,t}^o \: Q_t^o(B_{\bm{s}})
\]
}

\end{frame}

%==================================================================================================

\begin{frame}
\frametitle{Spatial Dependence}

Spatial dependence enters the model through the $\beta_{0,t}^i(s)$ parameters for $i \in \{o,1,2,3,4,5\}$.

\[ \beta_{0,t}^i(\bm{s}) = {\sigma}^i_t~w^i_t(\bm{s}) \]

\vspace{5mm}

where $w^i_{t}(\bm{s})$ are zero mean, variance $1$, Gaussian processes with exponential correlation given by 

\begin{align*}
\text{corr}(w^i_{t}(\bm{s}),w^i_{t}(\bm{s}')) = \exp(-\phi^i_{t} |\bm{s}-\bm{s}'|)
\end{align*}

Additional dependence between species is introduces via coregionalization,

\[
 \left( \begin{array}{cc} \beta^i_{0,t}(\boldsymbol{s})\\ \beta^j_{0,t}(\boldsymbol{s}) \end{array} \right)
 = \boldsymbol{A}_t \left( \begin{array}{cc} w^i_t(\boldsymbol{s})\\ w^j_t(\boldsymbol{s}) \end{array} \right).
\]

\end{frame}

%==================================================================================================

% \begin{frame}
% \frametitle{Model Details}

% For each species and network,

% \vspace{-5mm}

% \begin{columns}[t]
% \column{0.4\textwidth}
% \begin{align*}
% C_t^i(\bm{s}) &= Z_t^i(\bm{s}) + \epsilon_{C,t}^i(\bm{s}) \\
% I_t^i(\bm{s}) &= Z_t^i(\bm{s}) + \epsilon_{I,t}^i(\bm{s}) \\
% \end{align*}

% \column{0.6\textwidth}\pause
% \begin{align*}
% {Z}_t^i(\bm{s}) &= \max{}\left(0,~\widetilde{Z}_t^i(\bm{s})\right) \\
% \widetilde{Z}_t^i(\bm{s}) &= \beta_{0,t}^i +\beta_{0,t}^i(\bm{s}) + \beta_{1,t}^i \: Q_t^i(B_{\bm{s}})  
% \end{align*}
% \end{columns}

% \vspace{2mm} \pause

% For total PM,
% \begin{columns}[c]
% \column{0.42\textwidth}
% \begin{align*}
% C_t^{tot}(\bm{s}) &= Z_t^{tot}(\bm{s}) + \epsilon_{C,t}^{tot}(\bm{s}) \\
% I_t^{tot}(\bm{s}) &= Z_t^{tot}(\bm{s}) + \epsilon_{I,t}^{tot}(\bm{s}) \\
% F_t^{tot}(\bm{s}) &= Z_t^{tot}(\bm{s}) + \epsilon_{F,t}^{tot}(\bm{s})
% \end{align*}

% \column{0.58\textwidth} \pause
% \vspace{6mm}
% \[ Z^{tot}_{t}(\bm{s}) = \sum_{i=1}^{5} Z^i_t(\bm{s}) + Z^{o}_{t}(\bm{s}) \]
% \end{columns}

% \vspace{3mm} \pause

% \begin{align*}
% {Z}_t^o(s) &= \max{}\left(0,~\widetilde{Z}_t^o(s)\right) \\
% \widetilde{Z}_t^{o}(\bm{s}) &= \beta_{0,t}^o +\beta_{0,t}^o(\bm{s}) + \beta_{1,t}^o \: Q_t^o(B_{\bm{s}})
% \end{align*}

% \end{frame}

% %==================================================================================================

% \begin{frame}
% \frametitle{Model Details Cont.}

% Spatial dependence enters the model through the $\beta_{0,t}^i(s)$ and $\beta_{0,t}^o(s)$ parameters.


% \[ \beta_{0,t}^i(\bm{s}) = {\sigma}^i_t~w^i_t(\bm{s}) 
% \hspace{10mm}
% \beta_{0,t}^o(\bm{s}) = {\sigma}^o_t~w^o_t(\bm{s}) \]

% \vspace{5mm}

% where $w^i_{t}(\bm{s})$ and $w^o_{t}(\bm{s})$ are zero mean, variance $1$, Gaussian processes with exponential correlation given by 


% \begin{align*}
% \text{corr}(w^i_{t}(\bm{s}),w^i_{t}(\bm{s}')) = \exp(-\phi^i_{t} |\bm{s}-\bm{s}'|) \\
% \\
% \text{corr}(w^o_{t}(\bm{s}),w^o_{t}(\bm{s}')) = \exp(-\phi^o_{t} |\bm{s}-\bm{s}'|)
% \end{align*}


% \end{frame}

%==================================================================================================

\begin{frame}[label=pm_results]
\frametitle{Model results}

\vfill

\begin{center}
\only<1>{\includegraphics[width=0.8\textwidth]{figs/pm_maps.pdf}}
\only<2>{\includegraphics[width=\textwidth]{figs/pm_ts.pdf}\\
         \includegraphics[width=0.7\textwidth]{figs/pm_ts_legend.pdf}}
\end{center}

\vfill

\end{frame}

%==================================================================================================

\begin{frame}
\frametitle{MCMC performance}

\begin{center}
\renewcommand*{\arraystretch}{1.5}
\begin{tabular}{l|c|c|c}
Parameter                 & CPU (secs) & CPU+GPU (secs)  & Rel. Performance \\
\hline
$\beta_0, \; \beta_1$     & 0.00029    & 0.00030         & 0.97 \\
$\beta_0(s) $             & 0.09205    & 0.09132         & 1.00 \\
$\sigma^2$                & 0.00383    & 0.00385         & 0.99 \\
$\phi$                    & 0.46084    & 0.25174         & 1.83 \\
$\tau^2_i,\; \tau^2_{tot}$& 0.00003    & 0.00003         & 1.00 \\
\hline
Total                     & 0.55708    & 0.34729         & 1.60
\end{tabular}
\end{center}

\end{frame}

%==================================================================================================

\begin{frame}
\frametitle{Run times}

Total run time for model fitting (50,000 iterations): \\ \vspace{2mm}

\begin{columns}[c]
\column{0.5\textwidth}
\begin{itemize}
\item CPU - 7.7 hours
\item CPU+GPU - 4.8 hours
\end{itemize}
\column{0.5\textwidth}
$\times$ 52 weeks
\end{columns}


\vspace{7mm} \pause


Total run time for model prediction at 5950 locations (1,000 iterations): \\ \vspace{2mm}

\begin{columns}[c]
\column{0.5\textwidth}
\begin{itemize}
\item CPU - 7.2 hours
\item CPU+GPU - 4.3 hours
\end{itemize}
\column{0.5\textwidth}
$\times$ 52 weeks
\end{columns}

\vspace{7mm} \pause

One run takes about 775 hours (4.6 days) total on CPU alone, 473 (2.8 days) on CPU and GPU. 

\pause

\begin{center}
$\times$ 3 model variants\\
$\times$ 10 for cross validation
\end{center}

\end{frame}

%==================================================================================================

\begin{frame}[label=pm_lessons]
\frametitle{Lessons}


\begin{itemize}
\item Established infrastructure makes a huge difference in development time
\begin{itemize}
  \vspace{2mm} \item 1 hour to go from CPU implementation to CPU+GPU implementation 
  \vspace{2mm} \item Code shown previously is 2/3 of the changes necessary \hyperlink{code}{\beamerbutton{Code}}
\end{itemize}

\vspace{2mm}
\item In practice, was easier to run CPU only code across more servers (configuration time / effort)

\begin{itemize}
\vspace{2mm} \item Not possible (or at least easy) for models variants that are not independent in time.
\vspace{2mm} \item There will be $\sim20$ desktops with GPUs available in the department (available via Condor)
\end{itemize}

\vspace{2mm}

\item Rcpp Attributes offer huge advantages in development and deployment
\begin{itemize}
\vspace{2mm} \item Simplifies external dependencies (locally)
\vspace{2mm} \item Full compatibility is the goal for RcppGP
\end{itemize}

\end{itemize}

\end{frame}

%==================================================================================================

\section{GPUs and Low Rank Approximations}

%==================================================================================================

\begin{frame}
\frametitle{Low rank approximations}

For a Gaussian process

\[ Y(\bm{s}) = x(\bm{s})' \, \bm{\beta} + w(\bm{s}) + \epsilon, \quad \epsilon \sim N(0,~\tau^2 \, I) \]
%
\[ w(\bm{s}) \sim N(0,~\bm{C}(\bm{s})), \quad \bm{C}(\bm{s},\bm{s}')=\sigma\,\rho(\bm{s},\bm{s}'|\theta) \]

we can approximate $\bm{C}(\bm{s})$ with a low rank approximation with the form $\bm{U}\,\bm{S}\,\bm{V}'$ where $\bm{U}$ and $\bm{V}$ are $n \times k$ and $\bm{S}$ is $k \times k$.

~\\\pause

This allows for the use of the Sherman-Morrison-Woodbury formula for the inverse (and determinant),

\[
\bm{C}(\bm{s})^{-1} \approx
 \left(\bm{A} + \bm{U}\bm{S}\bm{V'}\right)^{-1} = 
\bm{A}^{-1} - \bm{A}^{-1} \bm{U} \left(\bm{S}^{-1}+\bm{V}' \bm{A}^{-1} \bm{U}\right)^{-1}\bm{V}' \bm{A}^{-1}.
\]


\end{frame}

%==================================================================================================

\begin{frame}
\frametitle{Gaussian Predictive Processes}

For a rank $k$ approximation,

\begin{itemize}
\item Pick $k$ knot locations $\bm{s}^*$
\item Calculate knot covariance ($\bm{C}(\bm{s}^*)$) and knot cross-covariance ($\bm{C}(\bm{s}^*)^{-1}$)
\item Approximate full covariance
\end{itemize}

\[\bm{C}(\bm{s}) \approx \bm{C}(\bm{s},\bm{s}^*) \, \bm{C}(\bm{s}^*)^{-1} \, \bm{C}(\bm{s}^*,\bm{s}).\]


\begin{itemize}
\item Systematically underestimates variance, inflates $\tau^2$. 

\item Modified predictive process corrects this using 

{\small
\[
\bm{C}(\bm{s}) \approx
\bm{C}(\bm{s},\bm{s}^*) \, \bm{C}(\bm{s}^*)^{-1} \, \bm{C}(\bm{s}^*,\bm{s}) + \text{diag}\Big(\bm{C}(\bm{s}) - \bm{C}(\bm{s},\bm{s}^*) \, \bm{C}(\bm{s}^*)^{-1} \, \bm{C}(\bm{s}^*,\bm{s})\Big).
\]
}
\end{itemize}

\vvfill

{\footnotesize
\begin{center}
Banerjee, Gelfand, Finley, Sang (2008) \quad Finley, Sang, Banerjee, Gelfand (2008)
\end{center}
}
\end{frame}

%==================================================================================================

\begin{frame}
\frametitle{Low Rank Approximations via Random Projections}

\begin{enumerate}
\item Starting with an $m \times n$ matrix $\bm{A}$.
\item Draw an $n \times k+p$ Gaussian random matrix $\bm{\Omega}$.
\item Form $\bm{Y} = \bm{A}\,\bm{\Omega}$ and compute its QR factorization $\bm{Y} = \bm{Q}\,\bm{R}$
%\item For $j=1,2,\ldots,q$ where q is integer power
%   Form Y􏱷 = AT Q and compute its QR factorization Y􏱷 = Q􏱷 R􏱷 j j−1 j jj
%   Form Yj = AQ􏱷j and compute its QR factorization Yj = Qj Rj
%   End
%\item $Q = Q_q$
\item Form the $k+p \times n$ matrix $\bm{B}=\bm{Q}'\,\bm{A}$.
\item Compute the SVD of the small matrix $\bm{B}$, $\bm{B} = \bm{\hat{U}}\,\bm{S}\,\bm{V}'$.
\item Form the matrix $\bm{U} = \bm{Q} \, \bm{\hat{U}}$.
\end{enumerate}

~\\ \pause

Resulting approximation has nicely bounded expected error,

\[ \text{E } \, \| \bm{A} - \bm{U}\bm{S}\bm{V}'\| \leq \left[1 + \frac{4\sqrt{k+p}}{p-1} \sqrt{\min(m,n)} \right] \sigma_{k+1}. \]

\vvfill

{\footnotesize
\begin{center}
Halko, Martinsson, Tropp (2011)
\end{center}
}
\end{frame}


%==================================================================================================

\begin{frame}
\frametitle{Random Matrix Low Rank Decompositions and GPs}

Preceeding algorithm can be modified slightly to take advantage of the positive definite structure of a covariance matrix.

\begin{enumerate}
\item Starting with an $n \times n$ covariance matrix $\bm{A}$.
\item Draw an $n \times k+p$ Gaussian random matrix $\bm{\Omega}$.
\item Form $\bm{Y} = \bm{A}\,\bm{\Omega}$ and compute its QR factorization $\bm{Y} = \bm{Q}\,\bm{R}$
\item Form the $k+p \times k+p$ matrix $\bm{B}=\bm{Q}'\,\bm{A} \, \bm{Q}$.
\item Compute the eigen decomposition of the small matrix $\bm{B}$, $\bm{B} = \bm{\hat{U}}\,\bm{S}\,\bm{\hat{U}}'$.
\item Form the matrix $\bm{U} = \bm{Q} \, \bm{\hat{U}}$.
\end{enumerate}

Once again we have a bound on the error,

\[
   \text{E } \| \bm{A} - \bm{Q}(\bm{Q}'\bm{A}\bm{Q})\bm{Q}'\| 
 = \text{E } \| \bm{A} - \bm{U}\bm{S}\bm{U}'\| 
\lessapprox c \cdot \sigma_{k+1}. 
\]

\vvfill

{\footnotesize
\begin{center}
Halko, Martinsson, Tropp (2011), Banerjee, Dunson, Tokdar (2012)
\end{center}
}

\end{frame}


%==================================================================================================

\begin{frame}[fragile]
\frametitle{Low Rank Approximations and GPUs}

Both predictive process and random matrix low rank approximations are good candidates for acceleration using GPUs.

\vspace{3mm}

\begin{itemize}
\item Both use Sherman-Woodbury-Morrison to calculate the inverse (involves matrix multiplication, addition, and a small matrix inverse).

\vspace{3mm}

\item Predictive processes involves several covariance matrix calculations (knots and cross-covariance) and a small matrix inverse.

\vspace{3mm}

\item Random matrix low rank involves a large matrix multiplication ($\bm{A}\bm{\Omega}$) and several small matrix decompositions (QR, eigen).

\vspace{3mm}

\item Functionality for both approaches included in current version of RcppGP (inv\_lr and
inv\_pp).

\end{itemize}

\end{frame}

%==================================================================================================




\begin{frame}
\frametitle{Matrix inverse (fixed rank, strong dependence)}

\vspace{-8mm}

\only<1>{
\begin{center}
\begin{knitrout}\footnotesize
\definecolor{shadecolor}{rgb}{0.969, 0.969, 0.969}\color{fgcolor}

{\centering \includegraphics[width=0.95\textwidth]{plots/unnamed-chunk-3-1} 

}



\end{knitrout}
\end{center}
}

\only<2>{
\begin{center}
\begin{knitrout}\footnotesize
\definecolor{shadecolor}{rgb}{0.969, 0.969, 0.969}\color{fgcolor}

{\centering \includegraphics[width=0.95\textwidth]{plots/unnamed-chunk-4-1} 

}



\end{knitrout}
\end{center}
}

\vspace{-10mm}
\[ n = 15000,\quad k=\{100,\ldots,4900\} \]

\end{frame}

%==================================================================================================

\begin{frame}
\frametitle{Matrix inverse (fixed rank, weak dependence)}

\vspace{-8mm}

\only<1>{
\begin{center}
\begin{knitrout}\footnotesize
\definecolor{shadecolor}{rgb}{0.969, 0.969, 0.969}\color{fgcolor}

{\centering \includegraphics[width=0.95\textwidth]{plots/unnamed-chunk-5-1} 

}



\end{knitrout}
\end{center}
}

\only<2>{
\begin{center}
\begin{knitrout}\footnotesize
\definecolor{shadecolor}{rgb}{0.969, 0.969, 0.969}\color{fgcolor}

{\centering \includegraphics[width=0.95\textwidth]{plots/unnamed-chunk-6-1} 

}



\end{knitrout}
\end{center}
}

\vspace{-10mm}
\[ n = 15000,\quad k=\{100,\ldots,4900\} \]

\end{frame}


%==================================================================================================

\begin{frame}
\frametitle{Rand. Matrix Low Rank Decompositions for Prediction}

\vspace{-3mm}

This approach can also be used for prediction, if we want to sample 
\[ \mathcal{N}(0,\bm{\Sigma}) \text{ with }\Sigma \approx \bm{U} \bm{S} \bm{U}' = (\bm{U} \bm{S}^{1/2} \bm{U}')(\bm{U} \bm{S}^{1/2} \bm{U}')'\] 
then $X_{pred} = (\bm{U}\, \bm{S}^{1/2}\,\bm{U}') \times \bm{Z}$ where $Z_i \sim \mathcal{N}(0,1)$.

\pause

\begin{center}
\includegraphics[width=0.6\textwidth]{figs/RandLRPred.png}
\end{center}

\vspace{-10mm}

\[ n=1000, \quad p=10000 \]

\begin{center}
{\footnotesize
Dehdari, Deutsch (2012)
}
\end{center}

\end{frame}

%==================================================================================================

\begin{frame}
\frametitle{Future Directions}

\begin{itemize}

\item Refinement of RcppGP

\vspace{1mm}
\begin{itemize}
  \vspace{1mm} \item Transition to header only implementation
  \vspace{1mm} \item Transparent GPU to CPU failover
  \vspace{1mm} \item Support for fixed error (instead of rank) random matrix low rank decomposition
  \vspace{1mm} \item Thinking about out-of-memory based approaches
\end{itemize}

\vspace{1mm}

\item Future of GPUs, CUDA, and Magma

\vspace{1mm}

\begin{itemize}
  \vspace{1mm} \item Single vs. Multi-GPU algorithms
  \vspace{1mm} \item Mixed precision algorithms
  \vspace{1mm} \item NVBLAS
  \vspace{1mm} \item Unified memory
  \vspace{1mm} \item cuSolver
\end{itemize}

\end{itemize}

\end{frame}


%==================================================================================================

\begin{frame}
\frametitle{Acknowledgments}

\begin{columns}[t]
\column{0.5\textwidth}
\textbf{Migratory Connectivity}
\vspace{5mm}
\begin{itemize}
  \item John Novembre - UChicago
  \vspace{3mm} \item Thomas Smith - UCLA
  \vspace{3mm} \item Kristen Ruegg - UCLA, UCSC
  \vspace{3mm} \item Center for Tropical Research, UCLA IoES
\end{itemize}

\column{0.5\textwidth}
\textbf{Speciated \PM}
\vspace{2.5mm}
\begin{itemize}
  \item Alan Gelfand - Duke
  \vspace{3mm} \item Dave Holland - EPA
  \vspace{3mm} \item Erin Schliep - Duke
\end{itemize}
\end{columns}

\vfill

\begin{center}
\begin{tabular}{ll}
RcppGP & \url{https://github.com/rundel/RcppGP} \\ 
Talk & \url{https://github.com/rundel/Presentations}
\end{tabular}
\end{center}

\end{frame}

%==================================================================================================

\appendix

%==================================================================================================

\begin{frame}

\end{frame}

%==================================================================================================


\begin{frame}
\frametitle{Another Approach}

bigGP is an R package written by Chris Paciorek, et al.

\begin{itemize}
\item Specialized implementation of LA operation for GPs
\item Designed to run on large super computer clusters
\item Uses both shared and distributed memory
\item Able to fit models on the order of $n = 65$k (32 GB Cov. matrix)
\end{itemize}


\begin{center}
\includegraphics[width=0.6\textwidth]{figs/Paciorek.pdf}
\end{center}

\end{frame}


%==================================================================================================


\begin{frame}[label=regions]
\frametitle{Regions}

\vfill
\begin{center}
\includegraphics[width=0.9\textwidth]{figs/pm_regions.pdf}
\end{center}
\vfill

\hyperlink{pm_results}{\beamerbutton{Back}}

\end{frame}

\end{document}
